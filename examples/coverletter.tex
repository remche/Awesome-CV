%!TEX TS-program = xelatex
%!TEX encoding = UTF-8 Unicode
% Awesome CV LaTeX Template for Cover Letter
%
% This template has been downloaded from:
% https://github.com/posquit0/Awesome-CV
%
% Authors:
% Claud D. Park <posquit0.bj@gmail.com>
% Lars Richter <mail@ayeks.de>
%
% Template license:
% CC BY-SA 4.0 (https://creativecommons.org/licenses/by-sa/4.0/)
%


%-------------------------------------------------------------------------------
% CONFIGURATIONS
%-------------------------------------------------------------------------------
% A4 paper size by default, use 'letterpaper' for US letter
\documentclass[11pt, a4paper]{awesome-cv}

% Configure page margins with geometry
\geometry{left=1.4cm, top=.8cm, right=1.4cm, bottom=1.8cm, footskip=.5cm}

% Specify the location of the included fonts
\fontdir[fonts/]

% Color for highlights
% Awesome Colors: awesome-emerald, awesome-skyblue, awesome-red, awesome-pink, awesome-orange
%                 awesome-nephritis, awesome-concrete, awesome-darknight
\colorlet{awesome}{awesome-orange}
% Uncomment if you would like to specify your own color
% \definecolor{awesome}{HTML}{CA63A8}

% Colors for text
% Uncomment if you would like to specify your own color
% \definecolor{darktext}{HTML}{414141}
% \definecolor{text}{HTML}{333333}
% \definecolor{graytext}{HTML}{5D5D5D}
% \definecolor{lighttext}{HTML}{999999}

% Set false if you don't want to highlight section with awesome color
\setbool{acvSectionColorHighlight}{true}

% If you would like to change the social information separator from a pipe (|) to something else
\renewcommand{\acvHeaderSocialSep}{\quad\textbar\quad}


%-------------------------------------------------------------------------------
%	PERSONAL INFORMATION
%	Comment any of the lines below if they are not required
%-------------------------------------------------------------------------------
% Available options: circle|rectangle,edge/noedge,left/right
\photo[circle,noedge,left]{./examples/profile}
\name{Rémi}{Cailletaud}
\position{Architecte système{\enskip\cdotp\enskip}DevOps}
\address{11 rue de l'église Notre Dame des vignes, 38360 Sassenage, France}

\mobile{(+33)676997145}
\email{remi.cailletaud@gmail.com}
\homepage{www.remche.org}
\github{remche}
% \linkedin{posquit0}
% \gitlab{gitlab-id}
% \stackoverflow{SO-id}{SO-name}
% \twitter{@twit}
% \skype{skype-id}
% \reddit{reddit-id}
% \extrainfo{extra informations}

%\quote{``Be the change that you want to see in the world."}


%-------------------------------------------------------------------------------
%	LETTER INFORMATION
%	All of the below lines must be filled out
%-------------------------------------------------------------------------------
% The company being applied to
\recipient
  {UNESS.fr}
  {Université Numérique En Santé et Sport.fr}
% The date on the letter, default is the date of compilation
\letterdate{\today}
% The title of the letter
\lettertitle{Candidature au poste d'architecte et administrateur système}
% How the letter is opened
\letteropening{Madame, Monsieur,}
% How the letter is closed
\letterclosing{Cordialement,}
% Any enclosures with the letter
\letterenclosure[Attaché]{Curriculum Vitae}


%-------------------------------------------------------------------------------
\begin{document}

% Print the header with above personal informations
% Give optional argument to change alignment(C: center, L: left, R: right)
\makecvheader[R]

% Print the footer with 3 arguments(<left>, <center>, <right>)
% Leave any of these blank if they are not needed
\makecvfooter
  {\today}
  {Rémi Cailletaud~~~·~~~Lettre de motivation}
  {}

% Print the title with above letter informations
\makelettertitle

%-------------------------------------------------------------------------------
%	LETTER CONTENT
%-------------------------------------------------------------------------------
\begin{cvletter}

\lettersection{À\ propos de moi}
	Ingénieur de formation, je suis diplômé d'EPITA, une école dont la méthode pédagogique favorise l'autonomie et le sens de l'inititative. Par la suite, j'ai effectué un Master 2 Recherche afin de me rapprocher du monde de l'Enseignement Supérieur et de la Recherche. Recruté au CNRS en 2006, j'ai pu développer des compétences et acquérir une grande expérience dans l'administration des systèmes et réseaux. Héritant d'un existant minimal, j'ai su mettre à profit mon intérêt pour les nouvelles technologies et proposer à mon laboratoire des infrastructures performantes et hautement disponibles. Durant ces années à la tête d'un service informatique, j'ai su être à l'écoute des besoins des utilisateurs, mais aussi être force de proposition. Passionné, je mets la veille technologique au coeur de mon travail, et prend le temps de me former pour acquérir de nouvelles compétences. 


\lettersection{Pourquoi l'UNESS.fr ?}
	Je baigne depuis petit dans le monde de l'enseignement et de la recherche. C'est tout naturellement que j'ai rejoint le CNRS, mais j'ai à présent envie de nouveaux défis. Travailler sur la plateforme SIDES dans le cadre de SIDES 3.0 et y developper mes compétences en big data et web sémantique en est un à la hauteur des mes ambitions. Mieux comprendre les besoins de la pédagogie numérique et savoir y répondre en est un autre.	
%Issu d'une famille de sportifs et d'enseignants, moi-même grand amateur et pratiquant très régulier, rejoindre l'UNESS.fr serait une belle satisfaction.

\lettersection{Pourquoi moi ?}
	En rejoignant l'UNESS.fr, je saurai faire profiter de ma grande expérience en architecture et en sécurité des systèmes, mais aussi de mon sens de l'intiative et de mon goût pour les nouvelles technologies. Ma formation de développeur associée à mon expertise des systèmes fait de moi un candidat à la croisée des deux mondes, dans la mouvance devops. Mon goût pour les nouvelles méthodes de collaboration et de développement (Git\{hub,lab\}, Kanban, intégration et déploiement continus, \ldots) me semble un véritable atout. De plus ma curiosité pour les technologies émergentes comme les chaînes de blocs et mon sens de l'initiative pourront être à l'origine de nouveaux projets.
	Enfin ma connaissance des infrastructures de l'UGA/GRICAD est un atout qui facilitera sûrement la prise en main et le développement de celle de l'UNESS.fr.
\end{cvletter}


%-------------------------------------------------------------------------------
% Print the signature and enclosures with above letter informations
\makeletterclosing

\end{document}
